\documentclass{beamer}

\mode<presentation> {

\usetheme{Madrid}
\usecolortheme{rose}
}

\usepackage{graphicx} 
\usepackage{booktabs} 

%----------------------------------------------------------------------------------------
%	TITLE PAGE
%----------------------------------------------------------------------------------------

\title[Linux Day 2014]{Package Manager e sistemi di init} 

\author{Simone Lombardi}
\institute[HCSSLUG] 
{
Universit\`a di Fisciano \\ 
\medskip
\textit{smlb@archlinux.info} 
}
\date{24 Ottobre 2014} 

\begin{document}

\begin{frame}
\titlepage 
\end{frame}

\begin{frame}
\frametitle{Overview} 
\tableofcontents 
\end{frame}

%----------------------------------------------------------------------------------------
%	PRESENTATION SLIDES
%----------------------------------------------------------------------------------------

%------------------------------------------------
\section{Prima Parte} 
%------------------------------------------------

\subsection{Package Manager} 

\begin{frame}
\frametitle{Introduzione sui package manager}
Un package manager \`e una serie di strumenti utilizzati per gestire in modo automatico ed intuitivo il software in una distribuzione GNU/Linux. Generalmente possiamo installare, rimuovere ed aggiornare i package (il metodo differisce da distro a distro). 
\end{frame}

%------------------------------------------------

\begin{frame}
	\frametitle{Funzioni dei package manager}
	I package manager:
	\begin{itemize}
		\item gestiscono le dipendenze (non tutti hanno questa funzione)
		\item verificano il checksum di ogni singolo pacchetto da installare, quindi ne controllano la correttezza e completezza
		\item verificano eventuali firme digitali, ovvero controllano la chiave GPG per verificare l'attendibilit\`a di un package
		\item permettono di effettuare operazioni sui pacchetti installati, come aggiornamenti e in casi particolari anche downgrade (avanzamento a versione precedente)
	\end{itemize}
\end{frame}

%------------------------------------------------

\begin{frame}
\frametitle{Suddivisione package manager}
\begin{itemize}
\item \textbf{apt, aptitude, dpkg}: Debian e derivate
\item \textbf{emerge, portage, ebuilds}: Gentoo e derivate
\item \textbf{abs, makepkg, pacman}: Arch Linux
\item \textbf{pkgtool}: Slackware
\item \textbf{yum, dnf, rpm}: Fedora e derivate
\item \textbf{zypper}: OpenSUSE
\item \textbf{xbps}: Void Linux
\item \textbf{PackageKit}
\end{itemize}
\end{frame}

%------------------------------------------------

\begin{frame}
\frametitle{Differenze fra package manager}
\begin{table}
\begin{tabular}{| l | c | r |}
\toprule
\textbf{Nome} & \textbf{Risolve dipendenze} & \textbf{Formato}\\
\midrule
apt             &  \checkmark    & .deb \\
emerge    & \checkmark     & LOLWUT \\
dpkg         & NO   & .deb \\
pacman   & \checkmark     & .pkg.tar.xz \\
yum          & \checkmark     & .rpm \\
pkgtool  & NO   & .txz \\
zypper      & \checkmark     & .tgz \\
\bottomrule
\end{tabular}
\caption{Comparison dei vari package manager}
\end{table}
\end{frame}

%------------------------------------------------
\section{Seconda Parte}
\subsection{Sistemi di init}
%------------------------------------------------

\begin{frame}
\frametitle{Introduzione all'init}
Nei sistemi UNIX, un init \`e il primo processo avviato durante il boot ed \`e un demone che verr\`a killato quando viene effettivamente spenta la macchina. L'init \`e avviato dal kernel: ha PID 1 e si occupa di gestire varie operazioni all'interno di un calcolatore:
\begin{itemize}
	\item Avviare demoni (e/o script dell'utente)
	\item Gestire/Verificare il runlevel di una macchina
	\item Effettuare check in fase di spegnimento e accensione
\end{itemize}
\end{frame}

%------------------------------------------------

\begin{frame}
\frametitle{Runlevel}
Controllano quali processi/servizi sono avviati automaticamente dall'init, abbiamo sette runlevel:
\begin{itemize}
	\item 0: Halt
	\item 1: Single-User Mode
	\item 2: Multi-User Mode
	\item 3: Multi-User con Network
	\item 4: Definito dall'utente
	\item 5: Avvia il sistema normalmente
	\item 6: reboot
\end{itemize}

\textbf{NB}: \textit{questa \`e la lista dei runlevel standard, possono anche differire.}
\end{frame}

%------------------------------------------------

\begin{frame}
\frametitle{Systemd}
\textbf{Systemd} sta sostituendo SysVinit, esso dispone di svariate caratteristiche:
\begin{itemize}
	\item Parallelizzazione: processi avviati tutti via sockets
	\item Unit Files: gestione di varie procedure tramite files con estensioni diverse (.service, .timer, .socket, .mount etc)
	\item Journaling: in formato binario, interrogabile con \textit{journalctl}
	\item Compatibilit\`a: GNU/Linux-only
	\item Troppe \textit{features} da inserire in una lista
\end{itemize}
\end{frame}

%------------------------------------------------

\begin{frame}
	\frametitle{OpenRC}
	\textbf{OpenRC}:
	\begin{itemize}
		\item Portabile su tutti gli OS
		\item Codici e configurazioni separate
		\item Ordinamento dell'avvio dei demoni automatico
	\end{itemize}	 
\end{frame}

% -----------------------------------------------
\begin{frame}
	\frametitle{SysVinit ed Upstart}
	\textbf{SysVinit}
	\begin{itemize}
		\item PID 1
		\item Lancia processi via /etc/inittab
		\item Script contenuti in /etc/rc.d/init.d/ o /etc/init.d/
	\end{itemize}	
\textbf{Upstart}
\begin{itemize}
	\item Avviato da super-user
	\item Gestisce i servizi \textit{critici} del sistema
	\item Se l'init \textit{muore}, c'\`e un kernel panic
\end{itemize}
\end{frame}
%------------------------------------------------

\begin{frame}
	\frametitle{Info}
	Queste slides sono realizzate con \LaTeX  e rilasciate sotto GFDL.
	I sorgenti risiedono su \href{https://github.com/smlb/LinuxDay2014-smlb}{Github}: sono ottenibili e modificabili liberamente.
\end{frame}
%------------------------------------------------
\begin{frame}
\Huge{\centerline{runlevel 0}}
\huge{\centerline{The End}}
\end{frame}

%----------------------------------------------------------------------------------------

\end{document} 